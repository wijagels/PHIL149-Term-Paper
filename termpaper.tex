\documentclass[a4paper,12pt]{article}
\usepackage[utf8]{inputenc}
\usepackage[english]{babel}

\usepackage[
  pdftitle={Corpse Disposal: A Cost-Benefit Analysis},
  pdfauthor={William Jagels},
  colorlinks=true,linkcolor=blue,urlcolor=blue,citecolor=black,bookmarks=true,
bookmarksopenlevel=2]{hyperref}

\usepackage[square]{natbib}

\usepackage{titlesec}
\titlelabel{\thetitle.\quad}


\begin{document}
\title{Corpse Disposal: A Cost-Benefit Analysis}
\author{William Jagels}
\date{\today}
\begin{titlepage}
  \clearpage
  \maketitle
  \thispagestyle{empty}
\end{titlepage}

\section{Introduction}
2,626,418 people died in the United States in 2014~\cite{mortality}, equivalent to a daily death rate of nearly 7,200.
Death care is a necessary service as at-home burials are not feasible in an increasingly urbanized country.
It is critical that the corpse disposal process be environmentally and economically friendly in order to improve the well-being of immediate and future generations.
This paper will outline a framework for regulatory decisions regarding death care and discuss a methodology for comparing options.

\section{Methodology}
\subsection{Cost-Benefit Analysis}
Cost-benefit analysis (CBA) is a good candidate for a discussion of death care as current corpse disposal methods are well-defined and have had sufficient time to be studied.
However, CBA tends to result in cold-hearted decisions that disregard religious or moral objections to policies.
As a result, CBA needs to be augmented in order to take into account public opinions.
The case of death care is especially controversial as any policies enforced by a government will disproportionately affect certain religious or cultural groups.
CBA also fails to take into account the fluidity of culture and religion over larger time intervals.
\subsection{Augmented Cost-Benefit Analysis}
In order to account for cultural and religious objections, we must graft new steps onto traditional cost-benefit analysis\@.
As CBA is a quantitative approach, it logically follows to continue to use numbers in our new methodology.
\subsubsection{Identify Groups}
For each candidate policy, we count the number of individuals who we expect to disagree with the candidate policy.
As a short example, a policy requiring taxes to be filed online would cause us to count the Amish population along with any other cultures or religions opposed to filling out online forms.
\subsubsection{Assess Severity}
Once we have accounted for all the groups opposed to a candidate policy, we assess the degree to which each group is opposed to the candidate policy.
The degree of opposition can be divided into three categories:
\begin{enumerate}
  \item Group is opposed to the candidate policy for economic reasons
  \item Group is fundamentally opposed to complying with the candidate policy
  \item Group is fundamentally opposed to the existence of the candidate policy
\end{enumerate}
The rationale for including category one is because we can simultaneously account for disproportionate affects on subsets of the population, something that CBA also fails to account for.
Category three opposition groups are the most worrisome as their well-being will be impacted even if they are excluded from the candidate policy, so we must pay special attention to them in our methodology.
The severity of a category three group's opposition can be used to quickly throw out a candidate policy that would prompt terrorist action, as terrorism is generally bad for the economy and well-being.
\subsubsection{Estimate Opposition Decay}
This step only applies to category two and three opposition groups.
In order to become more informed, we estimate the rate at which each opposition group is expected to shrink due to gradual acceptance of the candidate policy or other natural trends.
To compare the strength of a policy's opposition groups over time, we can loosely define a function.
$$\textrm{Opposition strength of Policy} = \int_{start}^{end} (\textrm{Pop.\ of opposing groups at time }t) dt$$
Where start is the time the policy is implemented and end is some arbitrary time in the future.
When comparing two policies, start and end must be the same in order for the strengths to be comparable.
By integrating over time, we favor a policy whose opposition will reduce quickly over a policy where opposition will remain steady.
Category 1, 2, and 3 may be compared separately as each group creates different issues, or they can be compared together with some modification.
We can assign coefficients $\beta, \delta$ to be multipliers for categories 2 \& 3 signifying that group 2 is $\beta$ times worse than category 1 and category 3 is $\delta$ times worse than group 1.
Naturally these coefficients must be assessed on a case-by-case basis.

\subsubsection{Assess Inequality}
In order to account for disproportionate effects of a candidate policy, we inspect category one groups.
Similar to our technique for category two and three, we give greater weight to candidate policies that favor equality in the long run.
In order to estimate the inequality produced by a candidate policy,


\nocite{*}

\bibliographystyle{ieeetr}
\bibliography{\jobname}
\end{document}
